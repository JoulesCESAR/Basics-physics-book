\chapter*{Introducción}

Se espera que al finalizar el contenido de este libro el estudiante de Física tenga la capacidad para: Comprender como se debe 
estudiar Física, realizar un plan de estudio y como resolver problemas.

\section*{Invitación a la Física}

\vspace{1.0cm}

\subsection*{¿Como estudiar Física?} 

Adoptese una aptitud positiva hacia la disciplina, teniéndose presente que se trata de la más importante rama de las Ciencias 
Naturales, y por ende es importante comprender sus conceptos y teorias.

\subsection*{Conceptos y principios de la Física}

En el proceso de aprendizaje es útil tomar cuidadosamente notas en clases y luego preguntar aquellos aspectos que se desea 
esclarecer en clases y trabajos en laboratorio, podrá complementar el estudio y ayudar a esclarecer algunos puntos.

\subsection*{Plan de estudio}

Es importante que se establezcla un plan de estudio regular, de preferencia del trabajo diario. Las clases adquirirán mayor 
significado si se lee e investiga por anticipado el tema de Física a tratar en clase.\\

\begin{tcolorbox}
En lugar de tener una sesión de estudio durante toda la noche para revisar los conceptos básicos y ecuaciones es mejor una buena 
noche de reposo.
\end{tcolorbox}

\subsection*{Problemas}


\textit{Cuando estas solucionando un problema, no te preocupes. Ahora, después de que has resuelto el problema es el momento de 
preocuparse.} \textbf{Richard Feynman}
\vspace{1.0cm}

Debe de tratarse de resolver el mayor número posible de problemas, pero antes de eso debe de comprenderse los principios, 
conceptos y leyes básicas de los fenómenos naturales que estan involucrados en el problema implicado. Trate de encontrar 
soluciones alternativas a los mismos problemas. Como sugerencia para resolver un problema planteado de Física se puede seguir los 
siguientes pasos:\\

\begin{enumerate}
\item Leáse el problema cuidadosamente y tranquilamente las veces que sean necesarias hasta estar seguro de la situación descrita 
y de lo que quiere encontrarse o resolver.

\item Realize (dibuje) una representación gráfica del problema con los rótulos de las cantidades físicas implicadas en el 
problema.

\item Cuando se encamine a lo que se pregunta, debe identificarse que principio o principios básicos están en la situación 
actuando.

\item Seleccione una relación que involucre los datos conocidos y las incognitas del problema para aplicarla.

\item Sustituya los valores numéricos dados con las unidades apropiadas dentro de la ecuación.

\item Obtengase el valor numérico de la incognita con su respectiva unidad de medida.

\item Revise con minusiosidad que todo el desarrollo sea lo más lógico posible y que su respuesta obtenida sea un valor 
físicamente coherente.
 
\end{enumerate}

\subsection*{Experimentos}

\textit{...lo que necesitamos es imaginación, pero la imaginación encorsetada en la terrible camisa de fuerza que es el 
conocimiento, que no importa cuán hermosa sea tu conjetura, no importa cuán inteligente seas, quién hiciese la conjetura o cómo 
se 
llame. Si no está de acuerdo con el experimento, está mal.} \textbf{Richard Feynman}

\vspace{1.0cm}

La Física que se fundamenta en evidencias y observaciones experimentales, que en vista de ellas pueden servir para probar o 
refutar ideas, teorias y modelos propuestos. Cuando los experimentos no son accesibles pueden ser imaginados, teorizados o 
simulados en una computadora.
